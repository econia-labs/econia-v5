% cspell:words semithick
\documentclass[twocolumn]{article}
\usepackage{amsmath}
\usepackage{pgfplots}
\pgfplotsset{compat=1.18}
\usetikzlibrary{intersections}

% Tangent lines on graph per https://tex.stackexchange.com/a/198046.
\makeatletter
\def\parsenode[#1]#2\pgf@nil{%
    \tikzset{label node/.style={#1}}
    \def\nodetext{#2}
}
\tikzset{
    add node at x/.style 2 args={
        name path global=plot line,
        /pgfplots/execute at end plot visualization/.append={
                \begingroup
                \@ifnextchar[{\parsenode}{\parsenode[]}#2\pgf@nil
            \path [name path global = position line #1-1]
                ({axis cs:#1,0}|-{rel axis cs:0,0}) --
                ({axis cs:#1,0}|-{rel axis cs:0,1});
            \path [xshift=1pt, name path global = position line #1-2]
                ({axis cs:#1,0}|-{rel axis cs:0,0}) --
                ({axis cs:#1,0}|-{rel axis cs:0,1});
            \path [
                name intersections={
                    of={plot line and position line #1-1},
                    name=left intersection
                },
                name intersections={
                    of={plot line and position line #1-2},
                    name=right intersection
                },
                label node/.append style={pos=1}
            ] (left intersection-1) -- (right intersection-1)
            node [label node]{\nodetext};
            \endgroup
        }
    }
}
\makeatother

\begin{document}

\title{Concentrated Liquidity Market Maker with Reinvesting and Decimalized Ticks}
\author{Econia Labs}
\date{}

\maketitle

\section{Constant product equations}

Let price $p$ represent the ratio of quote $q$ per base $b$:

\begin{equation} \label{eqn:price}
  p = \frac{q}{b}
\end{equation}

In a constant product automated market maker, the following invariant holds:

\begin{equation} \label{eqn:liquidity}
  bq = L^2
\end{equation}

Here, $L = \sqrt{bq}$ represents liquidity.

\section{Concentrated liquidity equations}

Concentrated liquidity effectively shifts a virtual constant product curve with base reserves $b_v$
and quote reserves $q_v$ such that real reserves are exhausted at the price endpoints of a position
in range $[p_l, p_h]$, where $l$ and $h$ denote the low and high bounds on range price.

At $p_l$ real reserves are held entirely in base ($b_{max}$) while at $p_h$ real reserves are held
entirely in quote ($q_{max}$), yielding:

\begin{equation}
  (b_r + b_{max})(q_r + q_{max}) = L_v^2 = b_v q_v
\end{equation}

\begin{tikzpicture}
  \begin{axis}[
      axis lines = left,
      xlabel = Base,
      ylabel = Quote,
      xmin = 0,
      xmax = 3,
      ymin = 0,
      ymax = 3,
      ytick=\empty,
      xtick=\empty,
      extra x ticks = {1.5},
      extra x tick labels = {$b_{max}$},
      extra y ticks = {1.5},
      extra y tick labels = {$q_{max}$},
      extra y tick style = { tick label style = {rotate = 90 }},
      tick style = {black, thick, major tick length = 7pt},
      % Tangent lines on graph per https://tex.stackexchange.com/a/198046.
      tangent/.style={
          add node at x={2}{[
                  sloped,
                  append after command = {
                      (\tikzlastnode.west) edge [semithick, magenta] (\tikzlastnode.east)
                    },
                  minimum width=2cm
                ]},
          add node at x={0.5}{[
                  sloped,
                  append after command = {
                      (\tikzlastnode.west) edge [semithick, magenta] (\tikzlastnode.east)
                    },
                  minimum width=2cm
                ]}
        }
    ]
    \addplot [
      domain = 0:5,
      samples = 100,
      color = green,
      thick,
      tangent,
    ] { 1 / x };
    \addlegendentry{Virtual reserves}
    \addplot [
      domain = 0:5,
      samples = 100,
      color = blue,
      thick,
    ] { 1 / (x + 0.5) - 0.5 };
    \addlegendentry{Real reserves}
    \addplot [
      domain = 1.5:2,
      samples = 100,
      color = gray,
      very thick,
      dotted,
    ] { x - 1.5 };
    \addplot [
      domain = 0:.5,
      samples = 100,
      color = gray,
      very thick,
      dotted,
    ] { x + 1.5 };
    \node [left]  at (2.125, 0.625) {$p_l$};
    \node [left]  at (0.75, 2) {$p_h$};
  \end{axis}
\end{tikzpicture}

\end{document}
