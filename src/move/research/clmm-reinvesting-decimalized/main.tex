\documentclass[table, twocolumn]{article}
\usepackage{amsmath}
\usepackage{pgfplots}
\usepackage{xcolor}
\pgfplotsset{compat=1.18}
\usetikzlibrary{arrows.meta}
\usetikzlibrary{intersections}

% Tangent lines on graph per https://tex.stackexchange.com/a/198046.
\makeatletter
\def\parsenode[#1]#2\pgf@nil{%
  \tikzset{label node/.style = {#1}}
\def\nodetext{#2}
}
\tikzset{add node at x/.style 2 args = {name path global = plot line,
      /pgfplots/execute at end plot visualization/.append = {\begingroup
          \@ifnextchar[{\parsenode}{\parsenode[]}#2\pgf@nil
          \path [name path global = position line #1-1]
          ({axis cs:#1,0}|-{rel axis cs:0,0}) --
          ({axis cs:#1,0}|-{rel axis cs:0,1});
          \path [xshift = 1pt, name path global = position line #1-2]
          ({axis cs:#1,0}|-{rel axis cs:0,0}) --
          ({axis cs:#1,0}|-{rel axis cs:0,1});
          \path [
            name intersections =
              {of = {plot line and position line #1-1}, name = left intersection},
            name intersections =
              {of = {plot line and position line #1-2}, name = right intersection},
            label node/.append style = {pos = 1}
          ] (left intersection-1) -- (right intersection-1)
          node [label node]{\nodetext};
          \endgroup
        }
    }
}
\makeatother


\begin{document}

\title{Concentrated Liquidity Market Maker with Reinvesting and Decimalized Ticks}
\author{Econia Labs}
\date{}

\maketitle

\section{Base, quote, and price}

Consider the trading pair \texttt{APT/USDC}, \texttt{APT} denominated in \texttt{USDC}:

\begin{table}[ht]
  \centering
  \rowcolors{1}{lightgray}{gray}
  \begin{tabular}{|c|c|c|}
    \hline
    \rowcolor{cyan}
    Term        & Notation & Example       \\
    Base asset  & $b$      & \texttt{APT}  \\
    Quote asset & $q$      & \texttt{USDC} \\
    \hline
  \end{tabular}
  \caption{Base and quote asset definitions}
  \label{tab:base-quote-definition}
\end{table}

Price is denoted as the amount of quote per base, for example 18.26 \texttt{USDC} per
\texttt{APT} at the time of this writing:

\begin{equation} \label{eqn:price}
  p = \frac{q}{b}
\end{equation}

\section{Constant product automated market makers}

A constant product automated market maker pools together base and quote reserves, with the spot
price of the pool defined as the ratio of quote to base per equation~\ref{eqn:price}.

During a swap, either base or quote is provided to the pool in exchange for the other asset,
resulting in a change to the spot price:

\begin{table}[ht]
  \centering
  \rowcolors{1}{lightgray}{gray}
  \begin{tabular}{|c|c|c|}
    \hline
    \rowcolor{cyan}
    Term           & Before swap & After swap \\
    Base reserves  & $b_0$       & $b_f$      \\
    Quote reserves & $q_0$       & $q_f$      \\
    Spot price     & $p_0$       & $p_f$      \\
    \hline
  \end{tabular}
  \caption{Spot price definitions}
  \label{tab:spot-before-after-swap}
\end{table}

\begin{equation}
  p_0 = \frac{q_0}{b_0},
  p_f = \frac{q_f}{b_f}
\end{equation}

Throughout a swap, the following constant product invariant holds:

\begin{equation} \label{eqn:liquidity}
  b q = L^2
\end{equation}

That is, liquidity $L = \sqrt{bq}$ is held constant during a swap, which means that the product of
base and quote is identical before and after the swap:

\begin{equation} \label{eqn:liquidity}
  b_0 q_0 = b_f q_f
\end{equation}

\begin{figure}[ht]
  \centering
  \begin{tikzpicture}
  \begin{axis}[
      axis lines = left,
      xlabel = Base reserves,
      ylabel = Quote reserves,
      xmin = 0,
      xmax = 3,
      ymin = 0,
      ymax = 3,
      ytick=\empty,
      xtick=\empty,
      extra x ticks = {0.5, 2},
      extra x tick labels = {$b_0$, $b_f$},
      extra y ticks = {2, 0.5},
      extra y tick labels = {$q_0$, $q_f$},
      tick style = {black, thick, major tick length = 7pt},
      % Tangent lines on graph per https://tex.stackexchange.com/a/198046.
      tangent/.style={add node at x={2}{[
                  sloped, minimum width = 75pt,
                  append after command =
                    {(\tikzlastnode.west) edge [thick] (\tikzlastnode.east)}
                ]},
          add node at x={0.5}{[
                  sloped, minimum width = 75pt,
                  append after command =
                    {(\tikzlastnode.west) edge [thick] (\tikzlastnode.east)}
                ]}
        }
    ]
    \addplot [
      domain = 0:5,
      samples = 100,
      color = blue,
      thick,
      tangent,
    ] {1 / x};
    \addlegendentry{$bq = L^2$}
    \node at (2, 0.5) [circle, fill, scale = 0.5] {};
    \node at (2.2, 0.8) [] {$p_f = -\frac{dq}{db}|_{b_f} = \frac{q_f}{b_f}$} ;
    \node at (0.5, 2) [circle, fill, scale = 0.5] {};
    \node at (0.5, 2) [right] {$p_0 = -\frac{dq}{db}|_{b_0} = \frac{q_0}{b_0}$} ;
    \draw [dashed] (0.5, 0) -- (0.5, 2);
    \draw [dashed] (0, 2) -- (0.5, 2);
    \draw [dashed] (2, 0) -- (2, 0.5);
    \draw [dashed] (0, 0.5) -- (2, 0.5);
    \draw [arrows = {-Latex[]}] (0.5, 0.125) -- (2, 0.125);
    \node at (1.25, 0.25) [] {$b_{in}$} ;
    \draw [arrows = {-Latex[]}] (0.125, 2) -- (0.125, 0.5);
    \node at (0.3, 1.25) [] {$q_{out}$} ;
  \end{axis}
\end{tikzpicture}

  \caption{Constant product swap sell}
  \label{fig:cpmm-swap-sell}
\end{figure}

Figure~\ref{fig:cpmm-swap-sell} corresponds to a swap sell, with $b_{in}$ of base provided to the
pool, and $q_{out}$ received in return, for an average execution price $p_{sell}$ of:

\begin{equation} \label{eqn:liquidity}
  p_{sell} = \frac{q_{out}}{b_{in}}
\end{equation}

Notably, this average execution price (the slope of the secant line between $p_0$ and $p_f$) lies
between the pool spot price before the swap $p_0$ and the pool spot price after the swap $p_f$, as
shown in figure~\ref{fig:cpmm-swap-sell-price}.

\begin{figure}[ht]
  \centering
  \begin{tikzpicture}
  \begin{axis}[
      axis lines = left,
      xlabel = Base reserves,
      ylabel = Quote reserves,
      xmin = 0,
      xmax = 3,
      ymin = 0,
      ymax = 3,
      ytick=\empty,
      xtick=\empty,
      extra x ticks = {0.5, 2},
      extra x tick labels = {$b_0$, $b_f$},
      extra y ticks = {2, 0.5},
      extra y tick labels = {$q_0$, $q_f$},
      tick style = {black, thick, major tick length = 7pt},
      % Tangent lines on graph per https://tex.stackexchange.com/a/198046.
      tangent/.style={add node at x={2}{[
                  sloped, minimum width = 75pt,
                  append after command =
                    {(\tikzlastnode.west) edge [thick] (\tikzlastnode.east)}
                ]},
          add node at x={0.5}{[
                  sloped, minimum width = 75pt,
                  append after command =
                    {(\tikzlastnode.west) edge [thick] (\tikzlastnode.east)}
                ]}
        }
    ]
    \addplot [
      domain = 0:5,
      samples = 100,
      color = blue,
      thick,
      tangent,
    ] {1 / x};
    \addlegendentry{$bq = L^2$}
    \node at (2, 0.5) [circle, fill, scale = 0.5] {};
    \node at (2.1, 0.3) [] {$p_f$} ;
    \node at (0.5, 2) [circle, fill, scale = 0.5] {};
    \node at (0.25, 2) [right] {$p_0$} ;
    \draw [arrows = {-Latex[]}] (0.5, 0.5) -- (2, 0.5);
    \node at (1.25, 0.25) [] {$b_{in}$} ;
    \draw [arrows = {-Latex[]}] (0.5, 2) -- (0.5, 0.5);
    \node at (0.3, 1.25) [] {$q_{out}$} ;
    \draw [thick, dashed] (0.5, 2) -- (2, 0.5);
    \node at (1.65, 1.35) [] {$p_{sell} = \frac{q_{out}}{b_{in}}$} ;
  \end{axis}
\end{tikzpicture}

  \caption{Constant product swap sell execution price}
  \label{fig:cpmm-swap-sell-price}
\end{figure}

Per the constant product invariant, the output amount of a swap buy or sell is a direct function of
input amount and initial reserves:

\begin{table}[ht]
  \centering
  \rowcolors{1}{lightgray}{gray}
  \begin{tabular}{|c|c|c|}
    \hline
    \rowcolor{cyan}
    Term  & Swap input & Swap output \\
    Base  & $b_{in}$   & $b_{out}$   \\
    Quote & $q_{in}$   & $q_{out}$   \\
    \hline
  \end{tabular}
  \caption{Swap input and output definitions}
  \label{tab:spot-before-after-swap}
\end{table}

That is, for a swap buy:

\begin{align} \label{eqn:b-out-simple}
  b_0 q_0               & = b_f q_f         \nonumber                              \\
  b_0 q_0               & = (b_0 - b_{out})(q_0 + q_{in}) \nonumber                \\
  b_0 q_0               & = b_0 q_0 + b_0 q_{in} - b_{out}(q_0 + q_{in}) \nonumber \\
  b_{out}(q_0 + q_{in}) & = b_{0} q_{in} \nonumber                                 \\
  b_{out}               & = \frac{q_{in} b_0}{q_{in} + q_0}
\end{align}

And for a swap sell:

\begin{align} \label{eqn:q-out-simple}
  b_0 q_0               & = b_f q_f         \nonumber                              \\
  b_0 q_0               & = (b_0 + b_{in})(q_0 - q_{out}) \nonumber                \\
  b_0 q_0               & = b_0 q_0 + b_{in} q_0 - q_{out}(b_0 + b_{in}) \nonumber \\
  q_{out}(b_0 + b_{in}) & = b_{in} q_0 \nonumber                                   \\
  q_{out}               & = \frac{b_{in} q_0}{b_{in} + b_0}
\end{align}

\section{Concentrated liquidity equations}

Concentrated liquidity effectively shifts a virtual constant product curve with base reserves $b_v$
and quote reserves $q_v$ such that real reserves are exhausted at the price endpoints of a position
in range $[p_l, p_h]$, where $l$ and $h$ denote the low and high bounds on range price.

At $p_l$ real reserves are held entirely in base ($b_{max}$) while at $p_h$ real reserves are held
entirely in quote ($q_{max}$), yielding:

\begin{equation} \label{eqn:liquidity-translation}
  (b_r + b_{max})(q_r + q_{max}) = L_v^2 = b_v q_v
\end{equation}

Equation~\ref{eqn:liquidity-translation} is plotted in figure~\ref{fig:liquidity-translation}.

\begin{figure}[ht]
  \centering
  % cspell:words semithick
\begin{tikzpicture}
  \begin{axis}[
      axis lines = left,
      xlabel = Base reserves,
      ylabel = Quote reserves,
      xmin = 0,
      xmax = 3,
      ymin = 0,
      ymax = 3,
      ytick=\empty,
      xtick=\empty,
      extra x ticks = {1.5},
      extra x tick labels = {$b_{max}$},
      extra y ticks = {1.5},
      extra y tick labels = {$q_{max}$},
      extra y tick style = {tick label style = {rotate = 90}},
      tick style = {thick, major tick length = 7pt},
      legend style = {fill = black, draw = gray},
      % Tangent lines on graph per https://tex.stackexchange.com/a/198046.
      tangent/.style={add node at x={2}{[
                  sloped, minimum width=2cm,
                  append after command =
                    {(\tikzlastnode.west) edge [semithick, magenta] (\tikzlastnode.east)}
                ]},
          add node at x={0.5}{[
                  sloped, minimum width=2cm,
                  append after command =
                    {(\tikzlastnode.west) edge [semithick, magenta] (\tikzlastnode.east)}
                ]}
        }
    ]
    \addplot [
      domain = 0:5,
      samples = 100,
      color = green,
      thick,
      tangent,
    ] {1 / x};
    \addlegendentry{Virtual reserves}
    \addplot [
      domain = 0:5,
      samples = 100,
      color = blue,
      thick,
    ] {1 / (x + 0.5) - 0.5};
    \addlegendentry{Real reserves}
    \addplot [
      domain = 1.5:2,
      samples = 100,
      color = gray,
      very thick,
      dotted,
    ] {x - 1.5};
    \addplot [
      domain = 0:.5,
      samples = 100,
      color = gray,
      very thick,
      dotted,
    ] {x + 1.5};
    \node [left] at (2.125, 0.625) {$p_l$};
    \node [left] at (0.75, 2) {$p_h$};
  \end{axis}
\end{tikzpicture}

  \caption{Liquidity translation}
  \label{fig:liquidity-translation}
\end{figure}

For a swap that does not move the price outside the specified range, the input and output amounts
for a feeless swap can thus be calculated using equations~\ref{eqn:q-out-simple}
and~\ref{eqn:b-out-simple} applied to virtual reserves $b_v$ and $q_v$:

\begin{equation}
  q_{out} = \frac{b_{in} q_{v, 0}}{b_{in} + b_{v, 0}}
\end{equation}

\begin{equation}
  b_{out} = \frac{q_{in} b_{v, 0}}{q_{in} + q_{v, 0}}
\end{equation}

\end{document}
