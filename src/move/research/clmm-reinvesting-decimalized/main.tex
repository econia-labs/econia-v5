\documentclass[table, twocolumn]{article}
\usepackage{amsmath}
\usepackage[acronym]{glossaries}
\usepackage{pgfplots}
\usepackage{xcolor}
\pgfplotsset{compat=1.18}
\usetikzlibrary{arrows.meta}
\usetikzlibrary{intersections}

% Dark mode.
\pagecolor{black}
\color{gray}

% Tangent lines on graph per https://tex.stackexchange.com/a/198046.
\makeatletter
\def\parsenode[#1]#2\pgf@nil{%
  \tikzset{label node/.style = {#1}}
\def\nodetext{#2}
}
\tikzset{add node at x/.style 2 args = {name path global = plot line,
      /pgfplots/execute at end plot visualization/.append = {\begingroup
          \@ifnextchar[{\parsenode}{\parsenode[]}#2\pgf@nil
          \path [name path global = position line #1-1]
          ({axis cs:#1,0}|-{rel axis cs:0,0}) --
          ({axis cs:#1,0}|-{rel axis cs:0,1});
          \path [xshift = 1pt, name path global = position line #1-2]
          ({axis cs:#1,0}|-{rel axis cs:0,0}) --
          ({axis cs:#1,0}|-{rel axis cs:0,1});
          \path [
            name intersections =
              {of = {plot line and position line #1-1}, name = left intersection},
            name intersections =
              {of = {plot line and position line #1-2}, name = right intersection},
            label node/.append style = {pos = 1}
          ] (left intersection-1) -- (right intersection-1)
          node [label node]{\nodetext};
          \endgroup
        }
    }
}
\makeatother


\newacronym{cpamm}{CPAMM}{Constant Product Automated Market Maker}
\newacronym{lp}{LP}{Liquidity Provider}

\title{Concentrated Liquidity Market Maker with Reinvesting and Decimalized Ticks}
\author{Econia Labs}
\date{}

\begin{document}

\maketitle

\section{Base, quote, and price}

Consider the trading pair \texttt{APT/USDC}, \texttt{APT} denominated in \texttt{USDC}, denoted per
table \ref{tab:base-quote-definition}.

\begin{table}[!htb]
  \centering
  \begin{tabular}{|c|c|c|}
    \hline \rowcolor{blue}
    Term        & Notation & Example       \\ \hline
    Base asset  & $b$      & \texttt{APT}  \\ \hline
    Quote asset & $q$      & \texttt{USDC} \\ \hline
  \end{tabular}
  \caption{Base and quote asset definitions}
  \label{tab:base-quote-definition}
\end{table}

Equation \ref{eqn:price} defines price as the amount of quote per base, for example 17.27
\texttt{USDC} per \texttt{APT} at the time of this writing.

\begin{equation} \label{eqn:price}
  p = \frac{q}{b}
\end{equation}

\section{Constant product automated market makers}

\subsection{Swap without fees}

A \gls{cpamm} pools together base and quote reserves, with the spot price of the pool defined as the
ratio of quote to base per equation \ref{eqn:price}.

During a swap, either base or quote is provided to the pool in exchange for the other asset,
resulting in a change to the spot price. Per table \ref{tab:spot-before-after-swap} and equation
\ref{eqn:price-before-after} spot prices are defined before and after in terms of current pool
reserves.

\begin{table}[!htb]
  \centering
  \begin{tabular}{|c|c|c|}
    \hline \rowcolor{blue}
    Term           & Before swap & After swap \\ \hline
    Base reserves  & $b_0$       & $b_f$      \\ \hline
    Quote reserves & $q_0$       & $q_f$      \\ \hline
    Spot price     & $p_0$       & $p_f$      \\ \hline
  \end{tabular}
  \caption{Spot price definitions}
  \label{tab:spot-before-after-swap}
\end{table}

\begin{equation} \label{eqn:price-before-after}
  p_0 = \frac{q_0}{b_0},
  p_f = \frac{q_f}{b_f}
\end{equation}

During a swap, a \gls{cpamm} maintains the invariant described in equation
\ref{eqn:cpamm-invariant}.

\begin{equation} \label{eqn:cpamm-invariant}
  b q = L^2
\end{equation}

That is, liquidity $L = \sqrt{bq}$ is held constant during a swap, which means that per equation
\ref{eqn:cpamm-invariant-as-products} the product of base and quote is equivalent before and after
the swap.

\begin{equation} \label{eqn:cpamm-invariant-as-products}
  b_0 q_0 = b_f q_f
\end{equation}

\begin{figure}[!htb]
  \centering
  \begin{tikzpicture}
  \begin{axis}[
      axis lines = left,
      xlabel = Base reserves,
      ylabel = Quote reserves,
      xmin = 0,
      xmax = 3,
      ymin = 0,
      ymax = 3,
      ytick=\empty,
      xtick=\empty,
      extra x ticks = {0.5, 2},
      extra x tick labels = {$b_0$, $b_f$},
      extra y ticks = {2, 0.5},
      extra y tick labels = {$q_0$, $q_f$},
      tick style = {thick, major tick length = 7pt},
      legend style = {fill = black, draw = gray},
      % Tangent lines on graph per https://tex.stackexchange.com/a/198046.
      tangent/.style={add node at x={2}{[
                  sloped, minimum width = 75pt,
                  append after command =
                    {(\tikzlastnode.west) edge [thick] (\tikzlastnode.east)}
                ]},
          add node at x={0.5}{[
                  sloped, minimum width = 75pt,
                  append after command =
                    {(\tikzlastnode.west) edge [thick] (\tikzlastnode.east)}
                ]}
        }
    ]
    \addplot [
      domain = 0:5,
      samples = 100,
      color = blue,
      thick,
      tangent,
    ] {1 / x};
    \addlegendentry{$bq = L^2$}
    \node at (2, 0.5) [circle, fill, scale = 0.5] {};
    \node at (2.2, 0.8) [] {$p_f = -\frac{dq}{db}|_{b_f} = \frac{q_f}{b_f}$} ;
    \node at (0.5, 2) [circle, fill, scale = 0.5] {};
    \node at (0.5, 2) [right] {$p_0 = -\frac{dq}{db}|_{b_0} = \frac{q_0}{b_0}$} ;
    \draw [dashed] (0.5, 0) -- (0.5, 2);
    \draw [dashed] (0, 2) -- (0.5, 2);
    \draw [dashed] (2, 0) -- (2, 0.5);
    \draw [dashed] (0, 0.5) -- (2, 0.5);
    \draw [arrows = {-Latex[]}] (0.5, 0.125) -- (2, 0.125);
    \node at (1.25, 0.25) [] {$b_{in}$} ;
    \draw [arrows = {-Latex[]}] (0.125, 2) -- (0.125, 0.5);
    \node at (0.3, 1.25) [] {$q_{out}$} ;
  \end{axis}
\end{tikzpicture}

  \caption{\gls{cpamm} swap sell}
  \label{fig:cpamm-swap-sell}
\end{figure}

As described in table \ref{tab:swap-in-out} and figure \ref{fig:cpamm-swap-sell}, a swap sell with
$b_{in}$ of base provided to the pool results in $q_{out}$ received in return. Per equation
\ref{eqn:cpamm-avg-execution-price}, this operation has average execution price $p_{sell}$.

\begin{table}[!htb]
  \centering
  \begin{tabular}{|c|c|c|}
    \hline \rowcolor{blue}
    Term  & Swap input & Swap output \\ \hline
    Base  & $b_{in}$   & $b_{out}$   \\ \hline
    Quote & $q_{in}$   & $q_{out}$   \\ \hline
  \end{tabular}
  \caption{Swap input and output definitions}
  \label{tab:swap-in-out}
\end{table}

\begin{equation} \label{eqn:cpamm-avg-execution-price}
  p_{sell} = \frac{q_{out}}{b_{in}}
\end{equation}

Notably, this average execution price (the slope of the secant line between $p_0$ and $p_f$ in
figure \ref{fig:cpamm-swap-sell-price}) lies between the spot price before and after the swap, $p_0$
and $p_f$, respectively.

\begin{figure}[!htb]
  \centering
  \begin{tikzpicture}
  \begin{axis}[
      axis lines = left,
      xlabel = Base reserves,
      ylabel = Quote reserves,
      xmin = 0,
      xmax = 3,
      ymin = 0,
      ymax = 3,
      ytick=\empty,
      xtick=\empty,
      extra x ticks = {0.5, 2},
      extra x tick labels = {$b_0$, $b_f$},
      extra y ticks = {2, 0.5},
      extra y tick labels = {$q_0$, $q_f$},
      tick style = {thick, major tick length = 7pt},
      legend style = {fill = black, draw = gray},
      % Tangent lines on graph per https://tex.stackexchange.com/a/198046.
      tangent/.style={add node at x={2}{[
                  sloped, minimum width = 75pt,
                  append after command =
                    {(\tikzlastnode.west) edge [thick] (\tikzlastnode.east)}
                ]},
          add node at x={0.5}{[
                  sloped, minimum width = 75pt,
                  append after command =
                    {(\tikzlastnode.west) edge [thick] (\tikzlastnode.east)}
                ]}
        }
    ]
    \addplot [
      domain = 0:5,
      samples = 100,
      color = blue,
      thick,
      tangent,
    ] {1 / x};
    \addlegendentry{$bq = L^2$}
    \node at (2, 0.5) [circle, fill, scale = 0.5] {};
    \node at (2.1, 0.3) [] {$p_f$} ;
    \node at (0.5, 2) [circle, fill, scale = 0.5] {};
    \node at (0.25, 2) [right] {$p_0$} ;
    \draw [arrows = {-Latex[]}] (0.5, 0.5) -- (2, 0.5);
    \node at (1.25, 0.25) [] {$b_{in}$} ;
    \draw [arrows = {-Latex[]}] (0.5, 2) -- (0.5, 0.5);
    \node at (0.3, 1.25) [] {$q_{out}$} ;
    \draw [thick, dashed] (0.5, 2) -- (2, 0.5);
    \node at (1.65, 1.35) [] {$p_{sell} = \frac{q_{out}}{b_{in}}$} ;
  \end{axis}
\end{tikzpicture}

  \caption{\gls{cpamm} swap sell execution price}
  \label{fig:cpamm-swap-sell-price}
\end{figure}

Given the constant product invariant from equations \ref{eqn:cpamm-invariant} and
\ref{eqn:cpamm-invariant-as-products}, the output amount of a swap buy or sell is a direct function
of input amount and initial reserves, as described in equation \ref{eqn:b-q-out-simple} (full
derivation at equation \ref{eqn:b-q-out-simple-derivation}).

\begin{align} \label{eqn:b-q-out-simple}
  b_{out} = \frac{q_{in} b_0}{q_{in} + q_0}, q_{out} = \frac{b_{in} q_0}{b_{in} + b_0}
\end{align}

\subsection{Swap with fees}

Consider a pool fee rate $f_p$ assessed on the output of swap, for example 30 basis points such that
$f_p = \frac{30}{10000}$. By assessing a fee on the output amount and reinvesting it in the pool,
spot price slippage decreases and liquidity increases for each swap.

For example figure \ref{fig:cpamm-swap-sell-with-fee} denotes a swap sell, where $f_p q_{out}$ is
deducted from quote proceeds and reinvested in the pool, thus increasing available liquidity.

\begin{figure}[!htb]
  \centering
  \begin{tikzpicture}
  \begin{axis}[
      axis lines = left,
      xlabel = Base reserves,
      ylabel = Quote reserves,
      xmin = 0,
      xmax = 3,
      ymin = 0,
      ymax = 3,
      ytick=\empty,
      xtick=\empty,
      extra x ticks = {0.5, 2},
      extra x tick labels = {$b_0$, $b_f$},
      extra y ticks = {2, 0.5},
      extra y tick labels = {$q_0$, $q_f$},
      tick style = {black, thick, major tick length = 7pt},
      % Tangent lines on graph per https://tex.stackexchange.com/a/198046.
      tangent/.style={add node at x={2}{[
                  sloped, minimum width = 75pt,
                  append after command =
                    {(\tikzlastnode.west) edge [thick] (\tikzlastnode.east)}
                ]},
          add node at x={0.5}{[
                  sloped, minimum width = 75pt,
                  append after command =
                    {(\tikzlastnode.west) edge [thick] (\tikzlastnode.east)}
                ]}
        }
    ]
    \addplot [
      domain = 0:5,
      samples = 100,
      color = blue,
      thick,
      tangent,
    ] {1 / x};
    \addlegendentry{Before reinvestment}
    \addplot [
      domain = 0:5,
      samples = 100,
      color = green,
      thick,
      tangent,
    ] {1.75 / x};
    \addlegendentry{After reinvestment}
    \node at (2, 0.5) [circle, fill, scale = 0.5] {};
    \node at (2, 0.875) [circle, fill, scale = 0.5] {};
    \node at (0.5, 2) [circle, fill, scale = 0.5] {};
    \draw [dashed] (0.5, 0) -- (0.5, 2);
    \draw [dashed] (0, 2) -- (0.5, 2);
    \draw [dashed] (2, 0) -- (2, 0.875);
    \draw [dashed] (0, 0.5) -- (2, 0.5);
    \draw [dashed] (0, 0.875) -- (2, 0.875);
    \draw [arrows = {-Latex[]}] (0.5, 0.125) -- (2, 0.125);
    \node at (1.25, 0.25) [] {$b_{in}$} ;
    \draw [arrows = {-Latex[]}] (0.125, 2) -- (0.125, 0.5);
    \draw [arrows = {-Latex[]}] (0.625, 0.5) -- (0.625, 0.875);
    \node at (0.3, 1.25) [] {$q_{out}$} ;
    \node at (0.9, 0.6875) [] {$f_p q_{out}$} ;
  \end{axis}
\end{tikzpicture}

  \caption{\gls{cpamm} swap sell with fee reinvestment}
  \label{fig:cpamm-swap-sell-with-fee}
\end{figure}

\subsection{Liquidity provider tokens}

As the amount of liquidity in the pool grows so does the share of each \gls{lp} who holds \gls{lp}
tokens in the pool, which represent a share of the pool's liquidity.

Notably, the first \gls{lp} in the pool sets the spot price via the amount of base and quote
contributed, and receives a number of tokens equal to the amount of provided liquidity.

For example consider the initial base and quote contributions in table \ref{tab:first-lp-amounts}.
Per equation \ref{eqn:price}, the initial spot price is then 4 as shown in equation
\ref{eqn:first-lp-spot-price}.

\begin{table}[!htb]
  \centering
  \begin{tabular}{|c|c|c|}
    \hline \rowcolor{blue}
    Asset & Notation & Amount \\ \hline
    Base  & $b$      & 5      \\ \hline
    Quote & $q$      & 20     \\ \hline
  \end{tabular}
  \caption{Base and quote provided by first \gls{lp}}
  \label{tab:first-lp-amounts}
\end{table}

\begin{equation} \label{eqn:first-lp-spot-price}
  p = \frac{20}{5} = 4
\end{equation}

Per equation \ref{eqn:cpamm-invariant}, the \gls{lp} then receives 10 \gls{lp} tokens as shown in
equation \ref{eqn:first-lp-tokens}.

\begin{equation} \label{eqn:first-lp-tokens}
  L = \sqrt{5 \cdot 20} = 10
\end{equation}

Thereafter any new \gls{lp} must provide base and quote proportional to the amount of base and
quote in the pool, receiving in exchange an amount of \gls{lp} tokens representing their pro rata
share of the pool's liquidity.

Continuing the example, as described in table \ref{tab:new-lp-tokens} and equation
\ref{eqn:new-lp-tokens}, if the spot price does not change a new \gls{lp} may contribute another 10
base and 40 quote, receiving 20 \gls{lp} tokens in return. At this point the pool has 15 base and 60
quote.

\begin{table}[!htb]
  \centering
  \begin{tabular}{|c|c|c|}
    \hline \rowcolor{blue}
    Term                           & Notation      & Amount \\ \hline
    Base in pool                   & $b_{pool}$    & 5      \\ \hline
    New base deposited             & $b_{deposit}$ & 10     \\ \hline
    Existing \gls{lp} token supply & $T_s$         & 10     \\ \hline
    New \gls{lp} tokens issued     & $T_n$         & 20     \\ \hline
  \end{tabular}
  \caption{Additional \gls{lp} token issuance definitions}
  \label{tab:new-lp-tokens}
\end{table}

\begin{equation} \label{eqn:new-lp-tokens}
  T_n = \frac{b_{deposit}}{b_{pool}} T_s = \frac{10}{5} 10 = 20
\end{equation}

As the pool processes trades and accumulates fees, the spot price may change and liquidity grows.
For example consider that over time the amount of liquidity in the pool increases and price drops by
50\%, yielding a pool with 30 base and 60 quote. Each \gls{lp} is still entitled to a pro rata
portion of the pool based on their share of total \gls{lp} tokens issued, so the first \gls{lp}
would receive 10 base and 20 quote if they were to burn their \gls{lp} tokens, while the second
\gls{lp} would receive 20 base and 40 quote.

Notably, Uniswap v2 burns the first 1000 \gls{lp} tokens (equivalent to asset subunits) to prevent
attacks that would enable the first contributor to a pool to make subsequent liquidity provisioning
prohibitively expensive.

\section{Concentrated liquidity equations}

Concentrated liquidity effectively shifts a virtual constant product curve with base reserves $b_v$
and quote reserves $q_v$ such that real reserves are exhausted at the price endpoints of a position
in range $[p_l, p_h]$, where $l$ and $h$ denote the low and high bounds on range price.

At $p_l$ real reserves are held entirely in base ($b_{max}$) while at $p_h$ real reserves are held
entirely in quote ($q_{max}$), yielding:

\begin{equation} \label{eqn:liquidity-translation}
  (b_r + b_{max})(q_r + q_{max}) = L_v^2 = b_v q_v
\end{equation}

Equation~\ref{eqn:liquidity-translation} is plotted in figure~\ref{fig:liquidity-translation}.

\begin{figure}[!htb]
  \centering
  % cspell:words semithick
\begin{tikzpicture}
  \begin{axis}[
      axis lines = left,
      xlabel = Base reserves,
      ylabel = Quote reserves,
      xmin = 0,
      xmax = 3,
      ymin = 0,
      ymax = 3,
      ytick=\empty,
      xtick=\empty,
      extra x ticks = {1.5},
      extra x tick labels = {$b_{max}$},
      extra y ticks = {1.5},
      extra y tick labels = {$q_{max}$},
      extra y tick style = {tick label style = {rotate = 90}},
      tick style = {thick, major tick length = 7pt},
      legend style = {fill = black, draw = gray},
      % Tangent lines on graph per https://tex.stackexchange.com/a/198046.
      tangent/.style={add node at x={2}{[
                  sloped, minimum width=2cm,
                  append after command =
                    {(\tikzlastnode.west) edge [semithick, magenta] (\tikzlastnode.east)}
                ]},
          add node at x={0.5}{[
                  sloped, minimum width=2cm,
                  append after command =
                    {(\tikzlastnode.west) edge [semithick, magenta] (\tikzlastnode.east)}
                ]}
        }
    ]
    \addplot [
      domain = 0:5,
      samples = 100,
      color = green,
      thick,
      tangent,
    ] {1 / x};
    \addlegendentry{Virtual reserves}
    \addplot [
      domain = 0:5,
      samples = 100,
      color = blue,
      thick,
    ] {1 / (x + 0.5) - 0.5};
    \addlegendentry{Real reserves}
    \addplot [
      domain = 1.5:2,
      samples = 100,
      color = gray,
      very thick,
      dotted,
    ] {x - 1.5};
    \addplot [
      domain = 0:.5,
      samples = 100,
      color = gray,
      very thick,
      dotted,
    ] {x + 1.5};
    \node [left] at (2.125, 0.625) {$p_l$};
    \node [left] at (0.75, 2) {$p_h$};
  \end{axis}
\end{tikzpicture}

  \caption{Liquidity translation}
  \label{fig:liquidity-translation}
\end{figure}

For a swap that does not move the price outside the specified range, the input and output amounts
for a feeless swap can thus be calculated using equation ~\ref{eqn:b-q-out-simple} applied to
virtual reserves $b_v$ and $q_v$:

\begin{equation}
  b_{out} = \frac{q_{in} b_{v, 0}}{q_{in} + q_{v, 0}},
  q_{out} = \frac{b_{in} q_{v, 0}}{b_{in} + b_{v, 0}}
\end{equation}

\appendix

\section{Derivations}

\begin{align} \label{eqn:b-q-out-simple-derivation}
  b_0 q_0               & = b_f q_f \nonumber                                      \\
  b_0 q_0               & = (b_0 - b_{out})(q_0 + q_{in}) \nonumber                \\
  b_0 q_0               & = b_0 q_0 + b_0 q_{in} - b_{out}(q_0 + q_{in}) \nonumber \\
  b_{out}(q_0 + q_{in}) & = b_{0} q_{in} \nonumber                                 \\
  b_{out}               & = \frac{q_{in} b_0}{q_{in} + q_0}, \nonumber             \\
  b_0 q_0               & = b_f q_f \nonumber                                      \\
  b_0 q_0               & = (b_0 + b_{in})(q_0 - q_{out}) \nonumber                \\
  b_0 q_0               & = b_0 q_0 + b_{in} q_0 - q_{out}(b_0 + b_{in}) \nonumber \\
  q_{out}(b_0 + b_{in}) & = b_{in} q_0 \nonumber                                   \\
  q_{out}               & = \frac{b_{in} q_0}{b_{in} + b_0}
\end{align}

\end{document}
