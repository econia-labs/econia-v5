\documentclass[twocolumn]{article}
\usepackage{amsmath}
\usepackage{pgfplots}
\pgfplotsset{compat=1.18}
\usetikzlibrary{intersections}

\begin{document}

\title{Concentrated Liquidity Market Maker with Reinvesting and Decimalized Ticks}
\author{Econia Labs}
\date{}

\maketitle

\section{Constant product equations}

Let price $p$ represent the ratio of quote $q$ per base $b$:

\begin{equation} \label{eqn:price}
  p = \frac{q}{b}
\end{equation}

In a constant product automated market maker, the following invariant holds:

\begin{equation} \label{eqn:liquidity}
  bq = L^2
\end{equation}

Here, $L = \sqrt{bq}$ represents liquidity, and is held constant during a swap.
Hence, for a known input amount of base $b_{in}$ or quote $q_{in}$, the corresponding output amount
of the other asset can be calculated as follows for initial base $b_0$ and quote $q_0$ (not
accounting for fees):

\begin{align} \label{eqn:q-out-simple}
  b_0 q_0               & = (b_0 + b_{in})(q_0 - q_{out}) \nonumber                \\
  b_0 q_0               & = b_0 q_0 + b_{in} q_0 - q_{out}(b_0 + b_{in}) \nonumber \\
  q_{out}(b_0 + b_{in}) & = b_{in} q_0 \nonumber                                   \\
  q_{out}               & = \frac{b_{in} q_0}{b_{in} + b_0}
\end{align}

\begin{align} \label{eqn:b-out-simple}
  b_0 q_0               & = (b_0 - b_{out})(q_0 + q_{in}) \nonumber                \\
  b_0 q_0               & = b_0 q_0 + b_0 q_{in} - b_{out}(q_0 + q_{in}) \nonumber \\
  b_{out}(q_0 + q_{in}) & = b_{0} q_{in} \nonumber                                 \\
  b_{out}               & = \frac{q_{in} b_0}{q_{in} + q_0}
\end{align}

\section{Concentrated liquidity equations}


Concentrated liquidity effectively shifts a virtual constant product curve with base reserves $b_v$
and quote reserves $q_v$ such that real reserves are exhausted at the price endpoints of a position
in range $[p_l, p_h]$, where $l$ and $h$ denote the low and high bounds on range price.

At $p_l$ real reserves are held entirely in base ($b_{max}$) while at $p_h$ real reserves are held
entirely in quote ($q_{max}$), yielding:

\begin{equation} \label{eqn:liquidity-translation}
  (b_r + b_{max})(q_r + q_{max}) = L_v^2 = b_v q_v
\end{equation}

Equation~\ref{eqn:liquidity-translation} is plotted in figure~\ref{fig:liquidity-translation}.

\begin{figure}[h]
  \centering
  % cspell:words semithick
\begin{tikzpicture}
  \begin{axis}[
      axis lines = left,
      xlabel = Base reserves,
      ylabel = Quote reserves,
      xmin = 0,
      xmax = 3,
      ymin = 0,
      ymax = 3,
      ytick=\empty,
      xtick=\empty,
      extra x ticks = {1.5},
      extra x tick labels = {$b_{max}$},
      extra y ticks = {1.5},
      extra y tick labels = {$q_{max}$},
      extra y tick style = {tick label style = {rotate = 90}},
      tick style = {thick, major tick length = 7pt},
      legend style = {fill = black, draw = gray},
      % Tangent lines on graph per https://tex.stackexchange.com/a/198046.
      tangent/.style={add node at x={2}{[
                  sloped, minimum width=2cm,
                  append after command =
                    {(\tikzlastnode.west) edge [semithick, magenta] (\tikzlastnode.east)}
                ]},
          add node at x={0.5}{[
                  sloped, minimum width=2cm,
                  append after command =
                    {(\tikzlastnode.west) edge [semithick, magenta] (\tikzlastnode.east)}
                ]}
        }
    ]
    \addplot [
      domain = 0:5,
      samples = 100,
      color = green,
      thick,
      tangent,
    ] {1 / x};
    \addlegendentry{Virtual reserves}
    \addplot [
      domain = 0:5,
      samples = 100,
      color = blue,
      thick,
    ] {1 / (x + 0.5) - 0.5};
    \addlegendentry{Real reserves}
    \addplot [
      domain = 1.5:2,
      samples = 100,
      color = gray,
      very thick,
      dotted,
    ] {x - 1.5};
    \addplot [
      domain = 0:.5,
      samples = 100,
      color = gray,
      very thick,
      dotted,
    ] {x + 1.5};
    \node [left] at (2.125, 0.625) {$p_l$};
    \node [left] at (0.75, 2) {$p_h$};
  \end{axis}
\end{tikzpicture}

  \caption{Liquidity translation}
  \label{fig:liquidity-translation}
\end{figure}






For a swap that does not move the price outside the specified range, the input and output amounts
for a feeless swap can thus be calculated using equations~\ref{eqn:q-out-simple}
and~\ref{eqn:b-out-simple} applied to virtual reserves $b_v$ and $q_v$:

\begin{equation}
  q_{out} = \frac{b_{in} q_{v, 0}}{b_{in} + b_{v, 0}}
\end{equation}

\begin{equation}
  b_{out} = \frac{q_{in} b_{v, 0}}{q_{in} + q_{v, 0}}
\end{equation}

\end{document}
